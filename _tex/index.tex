% Options for packages loaded elsewhere
\PassOptionsToPackage{unicode}{hyperref}
\PassOptionsToPackage{hyphens}{url}
\PassOptionsToPackage{dvipsnames,svgnames,x11names}{xcolor}
%
\documentclass[
]{agujournal2019}

\usepackage{amsmath,amssymb}
\usepackage{iftex}
\ifPDFTeX
  \usepackage[T1]{fontenc}
  \usepackage[utf8]{inputenc}
  \usepackage{textcomp} % provide euro and other symbols
\else % if luatex or xetex
  \usepackage{unicode-math}
  \defaultfontfeatures{Scale=MatchLowercase}
  \defaultfontfeatures[\rmfamily]{Ligatures=TeX,Scale=1}
\fi
\usepackage{lmodern}
\ifPDFTeX\else  
    % xetex/luatex font selection
\fi
% Use upquote if available, for straight quotes in verbatim environments
\IfFileExists{upquote.sty}{\usepackage{upquote}}{}
\IfFileExists{microtype.sty}{% use microtype if available
  \usepackage[]{microtype}
  \UseMicrotypeSet[protrusion]{basicmath} % disable protrusion for tt fonts
}{}
\makeatletter
\@ifundefined{KOMAClassName}{% if non-KOMA class
  \IfFileExists{parskip.sty}{%
    \usepackage{parskip}
  }{% else
    \setlength{\parindent}{0pt}
    \setlength{\parskip}{6pt plus 2pt minus 1pt}}
}{% if KOMA class
  \KOMAoptions{parskip=half}}
\makeatother
\usepackage{xcolor}
\setlength{\emergencystretch}{3em} % prevent overfull lines
\setcounter{secnumdepth}{5}
% Make \paragraph and \subparagraph free-standing
\makeatletter
\ifx\paragraph\undefined\else
  \let\oldparagraph\paragraph
  \renewcommand{\paragraph}{
    \@ifstar
      \xxxParagraphStar
      \xxxParagraphNoStar
  }
  \newcommand{\xxxParagraphStar}[1]{\oldparagraph*{#1}\mbox{}}
  \newcommand{\xxxParagraphNoStar}[1]{\oldparagraph{#1}\mbox{}}
\fi
\ifx\subparagraph\undefined\else
  \let\oldsubparagraph\subparagraph
  \renewcommand{\subparagraph}{
    \@ifstar
      \xxxSubParagraphStar
      \xxxSubParagraphNoStar
  }
  \newcommand{\xxxSubParagraphStar}[1]{\oldsubparagraph*{#1}\mbox{}}
  \newcommand{\xxxSubParagraphNoStar}[1]{\oldsubparagraph{#1}\mbox{}}
\fi
\makeatother


\providecommand{\tightlist}{%
  \setlength{\itemsep}{0pt}\setlength{\parskip}{0pt}}\usepackage{longtable,booktabs,array}
\usepackage{calc} % for calculating minipage widths
% Correct order of tables after \paragraph or \subparagraph
\usepackage{etoolbox}
\makeatletter
\patchcmd\longtable{\par}{\if@noskipsec\mbox{}\fi\par}{}{}
\makeatother
% Allow footnotes in longtable head/foot
\IfFileExists{footnotehyper.sty}{\usepackage{footnotehyper}}{\usepackage{footnote}}
\makesavenoteenv{longtable}
\usepackage{graphicx}
\makeatletter
\newsavebox\pandoc@box
\newcommand*\pandocbounded[1]{% scales image to fit in text height/width
  \sbox\pandoc@box{#1}%
  \Gscale@div\@tempa{\textheight}{\dimexpr\ht\pandoc@box+\dp\pandoc@box\relax}%
  \Gscale@div\@tempb{\linewidth}{\wd\pandoc@box}%
  \ifdim\@tempb\p@<\@tempa\p@\let\@tempa\@tempb\fi% select the smaller of both
  \ifdim\@tempa\p@<\p@\scalebox{\@tempa}{\usebox\pandoc@box}%
  \else\usebox{\pandoc@box}%
  \fi%
}
% Set default figure placement to htbp
\def\fps@figure{htbp}
\makeatother

\usepackage{url} %this package should fix any errors with URLs in refs.
\usepackage{lineno}
\usepackage[inline]{trackchanges} %for better track changes. finalnew option will compile document with changes incorporated.
\usepackage{soul}
\linenumbers
\makeatletter
\@ifpackageloaded{caption}{}{\usepackage{caption}}
\AtBeginDocument{%
\ifdefined\contentsname
  \renewcommand*\contentsname{Tabla de contenidos}
\else
  \newcommand\contentsname{Tabla de contenidos}
\fi
\ifdefined\listfigurename
  \renewcommand*\listfigurename{Listado de Figuras}
\else
  \newcommand\listfigurename{Listado de Figuras}
\fi
\ifdefined\listtablename
  \renewcommand*\listtablename{Listado de Tablas}
\else
  \newcommand\listtablename{Listado de Tablas}
\fi
\ifdefined\figurename
  \renewcommand*\figurename{Figura}
\else
  \newcommand\figurename{Figura}
\fi
\ifdefined\tablename
  \renewcommand*\tablename{Tabla}
\else
  \newcommand\tablename{Tabla}
\fi
}
\@ifpackageloaded{float}{}{\usepackage{float}}
\floatstyle{ruled}
\@ifundefined{c@chapter}{\newfloat{codelisting}{h}{lop}}{\newfloat{codelisting}{h}{lop}[chapter]}
\floatname{codelisting}{Listado}
\newcommand*\listoflistings{\listof{codelisting}{Listado de Listados}}
\makeatother
\makeatletter
\makeatother
\makeatletter
\@ifpackageloaded{caption}{}{\usepackage{caption}}
\@ifpackageloaded{subcaption}{}{\usepackage{subcaption}}
\makeatother

\ifLuaTeX
\usepackage[bidi=basic]{babel}
\else
\usepackage[bidi=default]{babel}
\fi
\babelprovide[main,import]{spanish}
% get rid of language-specific shorthands (see #6817):
\let\LanguageShortHands\languageshorthands
\def\languageshorthands#1{}
\usepackage{bookmark}

\IfFileExists{xurl.sty}{\usepackage{xurl}}{} % add URL line breaks if available
\urlstyle{same} % disable monospaced font for URLs
\hypersetup{
  pdftitle={Proyecto para el curso RESTful Web APIs con .NET Core},
  pdflang={es},
  colorlinks=true,
  linkcolor={blue},
  filecolor={Maroon},
  citecolor={Blue},
  urlcolor={Blue},
  pdfcreator={LaTeX via pandoc}}



\draftfalse

\begin{document}
\title{Proyecto para el curso RESTful Web APIs con .NET Core}

\authors{Kevin Andrés Hernández Rostrán\affil{1}}
\affiliation{1}{Universidad Cenfotec, San José, San Pedro, Costa Rica}








\section{Introducción}\label{introducciuxf3n}

A través de este proyecto, los estudiantes pondrán en práctica los
conocimientos adquiridos en el curso de desarrollo de APIs RESTful con
ASP.NET Core. Se creará un servicio web que servirá como punto de
partida para desarrollar aplicaciones más complejas y sofisticadas.

\subsection{Objetivos Básicos}\label{objetivos-buxe1sicos}

\begin{itemize}
\tightlist
\item
  \textbf{Crear una API RESTful básica:} Implementar los métodos HTTP
  fundamentales (GET, POST, PUT, DELETE) para realizar operaciones CRUD
  sobre un recurso sencillo (por ejemplo, una lista de tareas, una
  colección de usuarios).
\item
  \textbf{Utilizar ASP.NET Core:} Configurar un proyecto de ASP.NET
  Core, crear controladores, y modelar datos utilizando C\#.
\item
  \textbf{Entender el ciclo de vida de una solicitud HTTP:} Desde que se
  recibe una solicitud hasta que se envía una respuesta.
\item
  \textbf{Implementar enrutamiento básico:} Configurar rutas para
  diferentes operaciones sobre los recursos.
\end{itemize}

\subsection{Objetivos Intermedios}\label{objetivos-intermedios}

\begin{itemize}
\tightlist
\item
  \textbf{Implementar validación de datos:} Asegurarse de que los datos
  recibidos en las solicitudes sean válidos y consistentes.
\item
  \textbf{Manejar errores:} Implementar mecanismos para manejar errores
  y devolver respuestas informativas.
\item
  \textbf{Utilizar un ORM:} Interactuar con una base de datos utilizando
  un ORM como Entity Framework Core.
\item
  \textbf{Implementar autenticación básica:} Proteger la API utilizando
  mecanismos de autenticación sencillos como Basic Authentication o
  token-based authentication.
\end{itemize}

\subsection{Objetivos Avanzados}\label{objetivos-avanzados}

\begin{itemize}
\tightlist
\item
  \textbf{Implementar autorización:} Controlar el acceso a los recursos
  de la API en función de los roles de los usuarios.
\item
  \textbf{Utilizar middleware:} Crear middleware personalizado para
  realizar tareas comunes como la autenticación, la autorización, el
  logging o la compresión.
\item
  \textbf{Implementar paginación:} Manejar grandes conjuntos de datos de
  forma eficiente.
\item
  \textbf{Utilizar Swagger o OpenAPI:} Generar documentación interactiva
  de la API.
\end{itemize}

A continuación, se presentan varias propuestas de proyectos para que el
estudiante seleccione la que más se adapte a sus intereses y objetivos
de aprendizaje. Cada proyecto ofrece una oportunidad única para aplicar
los conocimientos adquiridos en el curso de una manera práctica y
creativa.

\section{Propuesta de proyecto A: Reverse Proxy con API de
Hacienda}\label{propuesta-de-proyecto-a-reverse-proxy-con-api-de-hacienda}

\subsection{Objetivo}\label{objetivo}

Desarrollar un reverse proxy utilizando ASP.NET Core que actúe como
intermediario entre una aplicación cliente y la API de Hacienda de Costa
Rica. Este proyecto permitirá a los estudiantes aplicar los
conocimientos adquiridos en el curso para construir una solución que
mejore el rendimiento, la seguridad y la escalabilidad de las
aplicaciones que interactúan con la API de Hacienda.

\subsection{Requisitos Mínimos}\label{requisitos-muxednimos}

\begin{itemize}
\item
  \textbf{Configuración del proxy:} Configurar el proxy para enrutar las
  solicitudes a los endpoints correspondientes de la API de Hacienda.
\item
  \textbf{Cacheo de respuestas:} Implementar un mecanismo de cacheo para
  almacenar las respuestas más comunes de la API y mejorar el
  rendimiento.
\item
  \textbf{Manejo de errores:} Implementar un manejo de errores robusto
  para capturar y gestionar las excepciones que puedan ocurrir durante
  las solicitudes.
\item
  \textbf{Autenticación:} Implementar un mecanismo de autenticación para
  acceder a la API de Hacienda de forma segura.
\end{itemize}

\subsection{Temas a Abordar}\label{temas-a-abordar}

\begin{itemize}
\item
  Aplicar los conceptos básicos de HTTP y desarrollo de APIs web para
  comprender el funcionamiento del proxy.
\item
  Utilizar los principios de REST para diseñar la arquitectura del proxy
  y las interacciones con la API de Hacienda.
\item
  Utilizar JSON para representar las respuestas de la API y para
  almacenar los datos en caché.
\item
  Implementar middleware para manejar la autenticación, el cacheo y el
  manejo de errores.
\item
  Diseñar una estructura de URL clara y consistente para los endpoints
  del proxy.
\item
  Generar una documentación completa de la API del proxy utilizando
  Swagger o OpenAPI.
\end{itemize}

\subsection{Requisitos Adicionales
(Opcionales)}\label{requisitos-adicionales-opcionales}

\begin{itemize}
\item
  \textbf{Compresión de respuestas:} Implementar la compresión de las
  respuestas para reducir el tamaño de los datos transmitidos.
\item
  \textbf{Seguridad:} Implementar medidas de seguridad adicionales, como
  el cifrado de las comunicaciones y la protección contra ataques
  comunes.
\item
  \textbf{Monitoreo:} Implementar un sistema de monitoreo para rastrear
  el rendimiento del proxy y detectar problemas.
\end{itemize}

\subsection{Tecnologías Sugeridas}\label{tecnologuxedas-sugeridas}

\begin{itemize}
\item
  \textbf{.NET Core:} Framework para el desarrollo del proxy.
\item
  \textbf{HttpClient:} Para realizar las solicitudes a la API de
  Hacienda.
\item
  \textbf{Redis:} Para implementar el cacheo.
\item
  \textbf{Swagger/OpenAPI:} Para generar la documentación de la API del
  proxy.
\end{itemize}

\subsection{Entrega}\label{entrega}

\begin{itemize}
\item
  Código fuente del proyecto.
\item
  Documentación de la API del proxy en formato Swagger o OpenAPI.
\item
  Un informe que describa la arquitectura del proxy, las decisiones de
  diseño y los desafíos enfrentados.
\item
  Diagrama de la arquitectura, mostrando los componentes principales:
  proxy, servicios externos, base de datos (opcional).
\end{itemize}

\subsection{Evaluación}\label{evaluaciuxf3n}

\begin{itemize}
\item
  Corrección de la implementación del proxy.
\item
  Eficiencia del cacheo y el manejo de errores.
\item
  Seguridad de la implementación.
\item
  Calidad del código y la documentación.
\end{itemize}

\subsection{Consideraciones
Adicionales}\label{consideraciones-adicionales}

\begin{itemize}
\item
  \textbf{Documentación de la API de Hacienda:} Es fundamental que los
  estudiantes consulten la documentación oficial de la API de Hacienda
  para entender los endpoints disponibles, los formatos de datos y los
  requisitos de autenticación.
\item
  \textbf{Performance:} Se debe prestar atención al rendimiento del
  proxy, especialmente en términos de tiempo de respuesta y uso de
  recursos.
\item
  \textbf{Escalabilidad:} El proxy debe ser diseñado para poder escalar
  horizontalmente si es necesario.
\end{itemize}

\subsection{Referencias}\label{referencias}

\begin{itemize}
\item
  \href{https://paper.dropbox.com/doc/API-Ministerio-de-Hacienda-znrOU6bGjTHcXjo8oUmBj}{API
  Ministerio de Hacienda}.
\item
  \href{https://medium.com/@kevinah95/hecho-en-costa-rica-hacienda-cli-d922490d6aca}{Hecho
  en Costa Rica: Hacienda CLI}.
\end{itemize}

\textbf{Al finalizar este proyecto, los estudiantes habrán adquirido
habilidades prácticas en el desarrollo de proxies, la integración con
APIs externas y la aplicación de los conceptos aprendidos en el curso.}

\section{Propuesta de proyecto B: API para Factura
Electrónica}\label{propuesta-de-proyecto-b-api-para-factura-electruxf3nica}

\subsection{Objetivo}\label{objetivo-1}

Desarrollar una API RESTful utilizando ASP.NET Core para gestionar la
emisión de facturas electrónicas. Esta API permitirá a las empresas
generar, consultar y cancelar facturas electrónicas de acuerdo con los
estándares establecidos por el Ministerio de Hacienda de Costa Rica. El
proyecto permitirá a los estudiantes aplicar los conocimientos
adquiridos en el curso para construir una solución que automatice y
optimice los procesos de facturación electrónica.

\subsection{Requisitos Mínimos}\label{requisitos-muxednimos-1}

\begin{itemize}
\item
  \textbf{Modelos de datos:} Definir modelos para representar facturas
  (número, fecha de emisión, cliente, productos, impuestos), clientes,
  productos y otros elementos relevantes.
\item
  \textbf{Endpoints:} Implementar endpoints RESTful para las siguientes
  operaciones:

  \begin{itemize}
  \tightlist
  \item
    \textbf{Facturas:} Crear, leer, actualizar, cancelar facturas.
  \item
    \textbf{Clientes:} Crear, leer, actualizar, eliminar clientes.
  \item
    \textbf{Productos:} Crear, leer, actualizar, eliminar productos.
  \item
    \textbf{Generación de XML:} Generar el archivo XML de la factura con
    el formato requerido por el Ministerio de Hacienda.
  \item
    \textbf{Firma digital:} Implementar la firma digital de la factura
    utilizando un certificado digital.
  \item
    \textbf{Envío al Ministerio de Hacienda:} Enviar la factura firmada
    al sistema del Ministerio de Hacienda.
  \end{itemize}
\item
  \textbf{Validación de datos:} Implementar una rigurosa validación de
  los datos de entrada para asegurar la integridad de las facturas.
\item
  \textbf{Manejo de errores:} Implementar un mecanismo de manejo de
  errores personalizado para devolver mensajes de error claros y
  concisos.
\item
  \textbf{Documentación:} Generar documentación de la API utilizando
  herramientas como Swagger o OpenAPI.
\end{itemize}

\subsection{Temas a Abordar}\label{temas-a-abordar-1}

\begin{itemize}
\tightlist
\item
  Aplicar los conceptos básicos de HTTP y desarrollo de APIs web para
  diseñar la arquitectura de la API.
\item
  Utilizar los principios de REST para diseñar los endpoints y las
  respuestas de la API.
\item
  Utilizar JSON para representar los datos y definir los formatos de
  respuesta.
\item
  Implementar middleware para tareas como la validación de datos, la
  autenticación, la autorización y el manejo de excepciones.
\item
  Diseñar una estructura de URL clara y consistente para los endpoints.
\item
  Generar una documentación completa de la API utilizando Swagger o
  OpenAPI.
\item
  Implementar un mecanismo de autenticación para proteger el acceso a la
  API.
\end{itemize}

\subsection{Requisitos Adicionales
(Opcionales)}\label{requisitos-adicionales-opcionales-1}

\begin{itemize}
\item
  \textbf{Generación de reportes:} Generar reportes de ventas y otros
  informes a partir de los datos de las facturas.
\item
  \textbf{Notificaciones:} Enviar notificaciones por correo electrónico
  o SMS cuando se emiten o cancelan facturas.
\end{itemize}

\subsection{Tecnologías Sugeridas}\label{tecnologuxedas-sugeridas-1}

\begin{itemize}
\item
  \textbf{.NET Core:} Framework para el desarrollo de aplicaciones web.
\item
  \textbf{Entity Framework Core:} ORM para interactuar con la base de
  datos.
\item
  \textbf{Swagger/OpenAPI:} Para generar la documentación de la API.
\item
  \textbf{Librería de firma digital:} Para firmar los archivos XML de
  las facturas.
\item
  \textbf{Biblioteca para consumir la API del Ministerio de Hacienda:}
  Para enviar las facturas al sistema del Ministerio.
\end{itemize}

\subsection{Entrega}\label{entrega-1}

\begin{itemize}
\item
  Código fuente del proyecto.
\item
  Documentación de la API en formato Swagger o OpenAPI.
\item
  Un documento explicativo que describa las decisiones de diseño y las
  tecnologías utilizadas.
\item
  Diagrama de la arquitectura, mostrando los componentes principales:
  api, servicios externos, base de datos (opcional).
\end{itemize}

\subsection{Evaluación}\label{evaluaciuxf3n-1}

\begin{itemize}
\item
  Corrección de la implementación de los endpoints.
\item
  Calidad del código (legibilidad, mantenibilidad).
\item
  Completitud de la documentación.
\item
  Cumplimiento de los requisitos adicionales.
\item
  Cumplimiento de los estándares de facturación electrónica del
  Ministerio de Hacienda.
\end{itemize}

\subsection{Consideraciones
Adicionales}\label{consideraciones-adicionales-1}

\begin{itemize}
\item
  \textbf{Legislación:} Es fundamental que los estudiantes se aseguren
  de cumplir con toda la legislación vigente en materia de facturación
  electrónica.
\item
  \textbf{Seguridad:} Se deben implementar medidas de seguridad robustas
  para proteger la información confidencial de los clientes y de la
  empresa.
\item
  \textbf{Performance:} La API debe ser eficiente y capaz de manejar un
  gran volumen de transacciones.
\end{itemize}

\subsection{Referencias}\label{referencias-1}

\begin{itemize}
\item
  \href{https://github.com/CRLibre/API_Hacienda}{API libre para Factura
  Electrónica en Costa Rica}.
\item
  \href{https://medium.com/@kevinah95/hecho-en-costa-rica-hacienda-cli-d922490d6aca}{Hecho
  en Costa Rica: Hacienda CLI}.
\item
  \href{https://raw.githubusercontent.com/CRLibre/docs-fe-hacienda-cr/master/diagrama-flujo/Diagrama\%20de\%20Flujo\%20para\%20Factura\%20Electronica\%20Costa\%20Rica.png}{Diagrama
  de flujo Factura Electrónica Costa Rica}.
\end{itemize}

\textbf{Al finalizar este proyecto, los estudiantes habrán adquirido una
sólida base en el desarrollo de APIs RESTful para aplicaciones
empresariales y estarán preparados para abordar proyectos más complejos
en el ámbito de la facturación electrónica.}

\section{Propuesta de proyecto C: Firmador
Electrónico}\label{propuesta-de-proyecto-c-firmador-electruxf3nico}

\subsection{Objetivo del Proyecto}\label{objetivo-del-proyecto}

Desarrollar una aplicación que permita firmar digitalmente documentos
electrónicos de forma segura y confiable. El firmador electrónico
garantizará la autenticidad, integridad y no repudio de los documentos
firmados.

\subsection{Lo que aprenderán los
estudiantes}\label{lo-que-aprenderuxe1n-los-estudiantes}

\begin{itemize}
\item
  \textbf{Criptografía:} Algoritmos de firma digital (RSA, DSA),
  generación de claves, certificados digitales.
\item
  \textbf{Seguridad de la información:} Protección de claves privadas,
  prevención de ataques.
\item
  \textbf{Desarrollo de aplicaciones:} Creación de interfaces de usuario
  intuitivas, manejo de archivos.
\item
  \textbf{Legislación:} Normativa relacionada con la firma electrónica.
\end{itemize}

\subsection{Criterios de Aceptación}\label{criterios-de-aceptaciuxf3n}

\begin{itemize}
\tightlist
\item
  \textbf{Funcionalidad:}

  \begin{itemize}
  \tightlist
  \item
    Firmar digitalmente diferentes tipos de archivos (XML, etc.).
  \item
    Verificar la integridad de una firma digital.
  \item
    Generar certificados digitales (opcional).
  \end{itemize}
\item
  \textbf{Seguridad:}

  \begin{itemize}
  \tightlist
  \item
    Almacenar las claves privadas de forma segura.
  \item
    Proteger contra ataques como la falsificación de firmas.
  \item
    Cumplir con los estándares de firma digital.
  \end{itemize}
\item
  \textbf{Usabilidad:}

  \begin{itemize}
  \tightlist
  \item
    Interfaz de usuario intuitiva y fácil de usar.
  \end{itemize}
\item
  \textbf{Integración:}

  \begin{itemize}
  \tightlist
  \item
    Posibilidad de integrarse con otros sistemas (opcional).
  \end{itemize}
\end{itemize}

\subsection{Descripción Detallada}\label{descripciuxf3n-detallada}

El firmador electrónico se construirá utilizando ASP .NET Core y se
desplegará en un ambiente local (no es necesario hostearlo en una nube).

\subsection{Retos y consideraciones}\label{retos-y-consideraciones}

\begin{itemize}
\item
  \textbf{Seguridad:} La seguridad de las claves privadas es
  fundamental.
\item
  \textbf{Legislación:} Es necesario cumplir con la legislación vigente
  en materia de firma electrónica.
\item
  \textbf{Interoperabilidad:} El firmador debe ser compatible con
  diferentes formatos de archivo y sistemas operativos.
\end{itemize}

\subsection{Temas a Abordar}\label{temas-a-abordar-2}

\begin{itemize}
\item
  Aplicar los conceptos básicos de HTTP para entender la comunicación
  entre el cliente y el servidor.
\item
  Utilizar los principios de REST para diseñar una API que permita
  firmar documentos de forma remota (opcional).
\item
  Utilizar JSON para representar los datos de la firma y los metadatos
  del documento.
\item
  Implementar middleware para manejar la autenticación, la autorización
  y el manejo de excepciones.
\item
  Diseñar una estructura de URL clara y consistente para los endpoints
  de la API (opcional).
\item
  Generar una documentación completa de la API utilizando Swagger o
  OpenAPI (opcional).
\item
  Implementar un mecanismo de autenticación robusto para proteger el
  acceso a las funcionalidades de firma.
\item
  Realizar pruebas exhaustivas para garantizar la seguridad y la
  funcionalidad del firmador.
\end{itemize}

\subsection{Aspectos adicionales a
considerar}\label{aspectos-adicionales-a-considerar}

\begin{itemize}
\item
  \textbf{Tipos de firma:} Firma simple, firma cualificada, firma
  avanzada.
\item
  \textbf{Integración con sistemas de gestión documental:} Para
  automatizar procesos de firma.
\item
  \textbf{Firma remota:} Permitir la firma de documentos de forma
  remota.
\end{itemize}

\subsection{Entrega}\label{entrega-2}

\begin{itemize}
\item
  Código fuente del proyecto.
\item
  Documentación de la API en formato Swagger o OpenAPI.
\item
  Un documento explicativo que describa las decisiones de diseño y las
  tecnologías utilizadas.
\item
  Diagrama de la arquitectura, mostrando los componentes principales:
  interfaz de usuario, motor de firma, almacenamiento de claves, etc.
\end{itemize}

\subsection{Referencias}\label{referencias-2}

\begin{itemize}
\item
  \href{https://github.com/CRLibre/API_Hacienda}{API libre para Factura
  Electrónica en Costa Rica}.
\item
  \href{https://medium.com/@kevinah95/hecho-en-costa-rica-hacienda-cli-d922490d6aca}{Hecho
  en Costa Rica: Hacienda CLI}.
\item
  \href{https://raw.githubusercontent.com/CRLibre/docs-fe-hacienda-cr/master/diagrama-flujo/Diagrama\%20de\%20Flujo\%20para\%20Factura\%20Electronica\%20Costa\%20Rica.png}{Diagrama
  de flujo Factura Electrónica Costa Rica}.
\item
  \href{https://github.com/johann04/xades-signer-cr}{xades-signer-cr}.
\end{itemize}

\textbf{Al finalizar este proyecto, los estudiantes habrán adquirido una
sólida base en el desarrollo de APIs RESTful para aplicaciones
empresariales y estarán preparados para abordar proyectos más complejos
en el ámbito de la facturación electrónica.}

\section{Propuesta de proyecto C: Reverse Proxy de Otras
APIs}\label{propuesta-de-proyecto-c-reverse-proxy-de-otras-apis}

\subsection{Objetivo del Proyecto}\label{objetivo-del-proyecto-1}

Desarrollar un reverse proxy utilizando ASP.NET Core que actúe como
intermediario entre una aplicación cliente y múltiples APIs externas.
Este proyecto permitirá a los estudiantes aplicar los conocimientos
adquiridos en el curso para construir una solución que consolide y
unifique el acceso a diferentes servicios web, mejorando la experiencia
del usuario y la mantenibilidad de las aplicaciones.

\subsection{Lo que aprenderán los
estudiantes}\label{lo-que-aprenderuxe1n-los-estudiantes-1}

\begin{itemize}
\item
  \textbf{Arquitectura de microservicios:} Cómo un proxy puede servir
  como puerta de entrada a múltiples servicios.
\item
  \textbf{Manejo de solicitudes HTTP:} Enrutamiento, transformación y
  composición de solicitudes.
\item
  \textbf{Caching:} Optimización del rendimiento a través del
  almacenamiento en caché de respuestas.
\item
  \textbf{Resiliencia:} Manejo de errores, timeouts y reintentos.
\item
  \textbf{Seguridad:} Autenticación, autorización y protección contra
  ataques comunes.
\end{itemize}

\subsection{Criterios de Aceptación}\label{criterios-de-aceptaciuxf3n-1}

\begin{itemize}
\tightlist
\item
  \textbf{Funcionalidad:}

  \begin{itemize}
  \tightlist
  \item
    Enrutar solicitudes a diferentes APIs basadas en la ruta de la URL.
  \item
    Agregar encabezados y parámetros de consulta a las solicitudes.
  \item
    Combinar datos de múltiples APIs en una sola respuesta.
  \end{itemize}
\item
  \textbf{Performance:}

  \begin{itemize}
  \tightlist
  \item
    Implementar un mecanismo de caching para mejorar la velocidad de
    respuesta.
  \item
    Manejar grandes volúmenes de tráfico.
  \end{itemize}
\item
  \textbf{Seguridad:}

  \begin{itemize}
  \tightlist
  \item
    Proteger contra ataques como inyección SQL, XSS y CSRF.
  \item
    Implementar autenticación y autorización para acceder a las APIs.
  \end{itemize}
\item
  \textbf{Escalabilidad:}

  \begin{itemize}
  \tightlist
  \item
    Diseñar el proxy para que pueda escalar horizontalmente.
  \end{itemize}
\end{itemize}

\subsection{Descripción Detallada}\label{descripciuxf3n-detallada-1}

El reverse proxy se construirá utilizando ASP.NET Core y se desplegará
en un ambiente local (no es necesario hostearlo en una nube).

\subsection{Tecnologías sugeridas}\label{tecnologuxedas-sugeridas-2}

\begin{itemize}
\item
  \textbf{.NET Core:} Framework para el desarrollo del proxy.
\item
  \textbf{HttpClient:} Para realizar las solicitudes a las APIs
  externas.
\item
  \textbf{Redis:} Para implementar el caching (opcional).
\item
  \textbf{Swagger/OpenAPI:} Para generar la documentación de la API del
  proxy (opcional).
\end{itemize}

\subsection{Temas a Abordar}\label{temas-a-abordar-3}

\begin{itemize}
\item
  Aplicar los conceptos básicos de HTTP para entender el funcionamiento
  del proxy.
\item
  Utilizar los principios de REST para diseñar los endpoints del proxy y
  las interacciones con las APIs externas.
\item
  Utilizar JSON para representar las respuestas de las APIs y para
  almacenar los datos en caché.
\item
  Implementar middleware para manejar la autenticación, el cacheo y el
  manejo de errores.
\item
  Diseñar una estructura de URL clara y consistente para los endpoints
  del proxy.
\item
  Generar una documentación completa de la API del proxy utilizando
  Swagger o OpenAPI.
\item
  Considerar el uso de Docker para contenerizar el proxy y facilitar su
  despliegue.
\end{itemize}

\subsection{Aspectos adicionales a
considerar}\label{aspectos-adicionales-a-considerar-1}

\begin{itemize}
\item
  \textbf{Seguridad:} Implementar medidas de seguridad adicionales, como
  el cifrado de las comunicaciones y la protección contra ataques
  comunes.
\item
  \textbf{Monitoreo:} Implementar un sistema de monitoreo para rastrear
  el rendimiento del proxy y detectar problemas.
\end{itemize}

\subsection{Referencias}\label{referencias-3}

\begin{itemize}
\item
  \href{https://auth0.com/blog/building-a-reverse-proxy-in-dot-net-core/}{Building
  a Reverse Proxy in .NET Core}.
\item
  \href{https://medium.com/@kevinah95/hecho-en-costa-rica-hacienda-cli-d922490d6aca}{Hecho
  en Costa Rica: Hacienda CLI}.
\item
  \href{https://dev.to/asimmon/how-to-securely-reverse-proxy-aspnet-core-web-apps-3h4c}{How
  to securely reverse-proxy ASP.NET Core web apps}.
\end{itemize}

\textbf{Al finalizar este proyecto, los estudiantes habrán adquirido
habilidades prácticas en el desarrollo de proxies, la integración con
APIs externas y la aplicación de los conceptos aprendidos en el curso.}




\end{document}
